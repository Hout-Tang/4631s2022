\documentclass{article}
\usepackage[
backend=biber,
style=alphabetic,
sorting=ynt
]{biblatex}
\usepackage{indentfirst}



\addbibresource{bibliography.bib}

\title{Implement and Design Agile methodology in mobile application development}
\author{Hout Tang}
\date{March 2, 2022}

\begin{document}


\maketitle

\section*{Abstract}
Every mobile software developer need an effective software development methodology to complete their mobile applications. There are two major famous methodologies in software developments such as agile and scrum methodologies. This paper will mainly discuss the challenges in mobile application development using agile and scrum methodologies. 


\section*{Introduction}
 In today’s world, almost every business is switching to mobile applications due to the growing number of digital gadgets. This is because businesses have understood the power of the internet and mobility in communication. Businesses find themselves in a digital environment with mobile applications that create the fastest marketing solutions for their products and services. In turn, the international market experiences demand in mobile applications serving different business interests. The situation has and continues to influence the development of mobile applications. One of the most significant design methodologies comprises agile and scrum processes. The agile methodology serves as an incremental process associated with mobile development. Some stages are involved to ensure the process is successful. Developers have ample opportunities to assess the possibility of apps moving in the right direction throughout developing them. The Scrum development process uses smaller modules to develop apps. It divides app development life cycles into bits to ensure that programming, development, and project management processes are efficient and successful in their application. Despite the scrum and agile methodologies remaining effective in mobile application development, they have several challenges. These challenges are complex and diverse such that they create some lagging outcomes in designing and creating mobile apps. The discussion evaluates these challenges for a comprehensive understanding of how they affect the mobile development environment. 


\section*{Challenges associated with Scrum and Agile Mobile Application Development}

Like any other process in mobile application development, challenges are imminent. In this case, the challenges associated with scrum and agile methodologies in a mobile application can be broken down in the following manner: 

\section*{Challenges of Agile Mobile Application Development Methodology}
Anyone who knows agile methodology can confirm that it is far from a silver bullet. This means that one cannot use it in mobile application development without having to comprehend its application. Also, its use without addressing the mobile development environment correctly will facilitate another failed transformation. Therefore, knowing the challenges is crucial in enabling the proper development of the mobile application environment. Agile Software Development Methodology suffers from Limited Predictability. It becomes difficult to quantify the required efforts when it comes to some software deliverable. This can be confirmed by starting the development life cycle on software projects that go beyond their normal size. Larger products require more effort. However, with agile software development, there are issues of failing to have the correct information on the magnitude of the process required to develop the right plan for completing larger projects. Therefore, it creates anxiety on teams that are new to agile methodology. The uncertainty that agile creates in product development on new teams can influence them to refuse to use it to meet their developmental needs \cite{tondel2019collaborative}. The fear such teams have can lead to constant frustrations when forced to use the method. It takes significant effort to understand how it works and its influence in developing competitive and real-time software projects for the market. Also, it can lead to poor practices and poor product development decisions. However, it becomes crucial for teams to regiment the agile development process to improve. This creates an opportunity to ease the quantification of cost, time, and effort required in delivering the final product. It demands more time, effort, motivation, and commitment

What is more annoying than dealing with the same methodology in completing a project? Individuals always hate to remain in the same cycle when they fail to develop a realistic project. In such instances, they suffer from anxiety and depression because they cannot move forward. In the case of agile methodology, commitment and investment in time and effort are crucial. Stakeholders such as testers, developers, and customers have to introduce efficient communication frameworks that influence constant interactions to achieve effectiveness in completing projects \cite{ciricReference}. Therefore, they must undertake numerous physical conversations to acquire information on the planning, designing, development, and testing process to ensure a successful software project. Close cooperation is an important approach that is required in the process. Also, daily users should avail themselves whenever they report problems with their mobile apps for prompt testing. The reason is that physically signing off on each developmental phase of apps created via agile processes is a major consideration. Apps developed via agile methodology must be remotely updated before integrating them with other features. Remote upgrades of apps are crucial to meet user expectations \cite{HODA201760}. However, this makes the process onerous and time-consuming. It also ensures individuals suffer from anxiety and burnout because of consuming too much time and energy to meet the needs of everyone involved. 
This leads to increased demands on both clients and developers
As reiterated above, agile requires constant communication and interaction with developers. Therefore, it demands close collaboration. Users have to be involved extensively for developers to know the issues connected to their mobile apps. Agile does not allow people to share information regarding an app and its features. The same is applicable when communicating about problems that need to be resolved with such apps. Therefore, the process includes agile methodology demanding big commitments regarding completing larger projects to succeed. Nevertheless, clients are forced to undergo training to assist developers in coming up with the best products \cite{dimaReference}. It means that any lack of client participation is detrimental to the quality and success of mobile apps. The other issue is that it will reflect poorly in the company that is involved in the process of developing the mobile app.It lacks documentation that is necessary for the design and development process. Documentation is important for the planning, design, development, and completion of mobile apps. However, with the digitization of mobile apps development, documentation is less detailed. The above reason also comes from the fact that mobile app’s requirements are clarified timely. The above shows that when new members are welcomed into a team, they may not gain the right knowledge about certain features associated with apps. The situation limits them from knowing what is required in the app’s development process, leading to difficulties and misunderstandings.Delays project success to influence projects falling off track.Mobile apps have to follow a specific path to achieve a successful developmental outcome. However, with agile methodology, little planning is required to start projects. Also, agile methodologies do not consider that consumer demands continue to change. Hence, once one begins a project, it may become difficult for them to have new inputs due to the specificity of agile methodology in the development process\cite{dimaReference}. The situation influences rigidity in encouraging innovation and creativity in the development of mobile apps. It means that in cases where there is a lack of communication, wrong areas of development may become the main issue. The situation can lead to mobile apps with many problems that cannot be corrected. 
\section*{Challenges with Scrum Mobile Application Methodologies }
Individuals have always found it easy to adopt the Scrum methodology for mobile development outcomes. However, several issues are associated with the methodology regarding the mobile application development process. 

\section*{Scrum methodology disrupts the operations of teams}
The scrum process in project management has to be evaluated comprehensively because it does not bring out the required results. The reason is that it takes significant time and effort to ensure that individuals can understand different components associated with the projects they undertake within their environments. The methodology entails laying out processes and working hard to ensure that they deliver the right mobile app. It does not encourage individuals to have the right information regarding the mobile app process development framework. Therefore, teamwork is almost impossible when it comes to using scrum methodology. In cases where product owners try to introduce new requirements, the team cannot interfere too much. The product owners can decide how the mobile app will have to be developed and can limit the developer to come up with inputs. Such a situation can create problems in developing the best features in a mobile app. If the project reaches the middle phase, disruptions from the product owner can create problems in the development process. Therefore, it becomes difficult for a development team to protect developing the mobile app\cite{Bhavsar_ScrumReference}. To achieve the right outcome, such teams have to make extra requests and add to the mobile app owners. The fact that teams are micro-managed is a major issue that needs to be changed at all times. Teams have to be left to imagine the best features they have to introduce in developing the right mobile apps.

\section*{Scrum methodology lacks adequacy in sprint duration}
Developing mobile applications requires individuals to come up with a stronger plan that is meant to ensure that they develop an efficient app. A scrum master has the power to establish the sprint duration through expert knowledge. Time management skills are important to ensure that sprints associated with mobile app developments are on track. However, this is not always the case with achieving the right outcomes to coming with apps \cite{karabulutarticle}. Issues such as time boxing may become a problem when dealing with scrum methodology. A timebox can delay a project by limiting the successful operations of some parts associated with the app development process. The reason is that teams can forget that they have to go through all stages simultaneously to know the gaps with them. The situation influences a mandatory breaking down of tasks into manageable chunks. 
On top of that, poor knowledge, skills, and Training Outcomes are also related.Knowledge and skills are important components of any development of a project. The main reason is that they offer the right guidance on dealing with gaps and developing the best steps in coming up with the right mobile apps for the market. However, scrum has complexities in terms of applying it in developing apps. In line with the above, not every organization invests heavily in training their employees to achieve a positive developmental outcome in the mobile apps field. Failure to train scrum practitioners means that projects cannot achieve their successful outcomes.The reason is that team members will only choose the relevant parts associated with the development process and leave some other parts unattended. The situation can create loopholes and gaps in the efficiency of mobile apps \cite{alqudah2016review}. The other outcome that is experienced may include a company having to hire other employees to deal with such problems. Therefore, it is crucial for everyone working on a project to know details associated with project management development. The right training creates an opportunity for individuals to work well in product development teams. It means that it gives them a competitive edge in defining how they can develop projects and meet their operational needs. 

\section*{Issues with Backlog Management}

Due to failure to receive the right training and poor knowledge of the mobile application development process. Scrum frameworks of mobile development tend to be immovable. These include the daily scrum and roles associated with the scrum master. The situation means that team members are left to handle issues by themselves. The outcome is that they are left with backlogs because they fail to complete subsequent phases associated with the developmental product process. Also, the process comes with many prioritized items associated with backlog but lacks the right guidance on how to complete them. Therefore, individuals are forced to learn more about such items to complete projects. 



\section*{Conclusions}
The discussion has shown the sensitivity of the mobile app developmental process via the use of agile and scrum methodologies. Despite the advantages of using agile and scrum methodologies, the paper has provided several challenges that have to be evaluated to prevent risks that affect the efficiency of mobile apps. Concerning agile methodology, several issues have been discussed. Agile methodology has limited predictability, demands time and commitment from stakeholders, increases demand from both clients and developers, and delays projects. Failure to use documentation in agile methodology is an issue for efficiency in the mobile app’s development. At the same time, the discussion has touched on the inefficiency of using scrum methodology in mobile apps development. The first challenge is that it causes disruption when it comes to the efficiency of teams in any given assignment. Secondly, it does not have adequacy in sprint duration leading to inefficient development of mobile apps. Failure to train employees leads to poor knowledge and skills, affecting how mobile apps are developed. It also ensures that individuals have problems with managing backlogs in mobile apps during the development process. Therefore, these challenges have to be considered to ensure efficiency in the development of mobile apps. 

%used to skip line 
\medskip

\printbibliography

\end{document}
