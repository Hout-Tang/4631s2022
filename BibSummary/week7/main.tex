\documentclass{article}
\usepackage[
backend=biber,
style=alphabetic,
sorting=ynt
]{biblatex}
\usepackage{indentfirst}



\addbibresource{bibliography.bib}

\title{Bibliography Summary}
\author{Hout Tang}
\date{March 16, 2022}

\begin{document}


\maketitle

\section*{Summary 1}
% First summary: 


React Native is programming framework which has been developed by Facebook. It offers developers the full ability to create native mobile applications for both iOS and Android. It uses the universal programming language called JavaScript \cite{BetsolArticle}. There are several advantages of using react native. The first advantage is saves time and money. Due to the fact that over 95 percents of code is cross-platform, developer needs to maintain two code bases for iOS and Android. With react native, developer only need to build one react native code base, and deploy to the platforms. The second advantage is that it provides great performance. React Native apps' performances are exactly the same as native app \cite{clearTeachArticle}.The final advantage is that the publishing updates for apps is much faster since React native streamlines the process of build process. 


%used to skip line 
\medskip

\section*{Summary 2}
% Second summary: 

Flutter is the cross-platform applications development instrument \cite{AnnaDziubaArticle}. It is similar to React Native which offers one code base and the developers can produce both iOS and Android development. There are several advantages of using Flutter. The first advantage is offering same User Interface and Business Logic in all platforms. The second advantage is that it reduces code development time because of its hot reload feature. The third advantage is that it increases the time-to-market speed. The flutter development framework functions quicker than alternatives. \cite{AnnaDziubaArticle}


\printbibliography

\end{document}
