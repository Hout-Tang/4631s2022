\documentclass{article}
\usepackage[
backend=biber,
style=alphabetic,
sorting=ynt
]{biblatex}
\usepackage{indentfirst}



\addbibresource{bibliography.bib}

\title{Bibliography Summary}
\author{Hout Tang}
\date{February 16, 2022}

\begin{document}


\maketitle

\section*{Summary 1}
% First summary: 
"Scrum is an agile development methodology which is used in the development of software based on an iterative and incremental processes."\cite{digiteArticle}. Scrum is also a fast, flexible and effective agile framework which is designed to provide or deliver value to customer throughout the development of the project \cite{digiteArticle}.In mobile application development sectors, agile with its Scrum and Kanban sub-types is the most popular framework and over 71 percents of the companies are reported to use them \cite{olgaArticle}. There are three major advantages of using scrum in mobile app development. Those are improving time management, increasing adaptability, promoting teamwork \cite{olgaArticle}.

%used to skip line 
\medskip

\section*{Summary 2}
% Second summary: 

The purpose of scrum artifacts are designed to ensure the transparency of key information in all decision making in mobile app development. Scrum artifacts consists of product backlog, sprint backlog, increment. Product backlog aka PB is a list which contains all product needs to satisfy the customers. it normally is prepared by the product owners. The goal of PB is to answer the question " What should be done "\cite{digiteArticle}. The sprint backlog is the subset of product backlog. It is performed during each Sprint. The Scrum board will be displayed during Sprint and it makes the development process visible to every team member who involves in the mobile development area. In addition, the increment is the summary of all tasks, product backlogs, use cases, user experiences and any elements that were developed during the sprint and will likely be made available to the end user in the form of software \cite{srivastava2017scrum}.




\printbibliography

\end{document}
